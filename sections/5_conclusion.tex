\radcbm turns routine radiology reports into large-scale concept supervision and aligns model reasoning with how radiologists read chest X-rays.
Automated concept extraction plus a hierarchical, gated CBM yields faithful, region-aware explanations without sacrificing classification accuracy, and the anatomy-first structure ensures that explanations are expressed in terms that are already familiar to clinicians.
Experiments on MIMIC-CXR and CheXpert show that the hierarchy improves concept fidelity over flat CBMs and matches black-box performance while exposing actionable concept interventions that allow users to probe and edit model behavior at the level of named findings.

Several avenues remain for future work.
First, the current vocabulary focuses on chest radiography; extending the concept hierarchy and extraction pipeline to CT, MRI, and multi-view studies will require modality-specific concept vocabularies and 3D region hierarchies for broader clinical impact.
Second, while \radcbm\ filters to visual, image-evident concepts, incorporating explicit uncertainty estimation for rare findings and out-of-distribution patterns may further improve safety in deployment.
Finally, prospective user studies with radiologists and other clinicians are needed to quantify how region-aware concept explanations affect trust, diagnostic decision-making, and workflow efficiency when integrated into real reporting environments.
